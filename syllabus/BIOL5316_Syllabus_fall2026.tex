\documentclass[12pt, notitlepage]{article}   	% use "amsart" instead of "article" for AMSLaTeX format
\usepackage{geometry}                		% See geometry.pdf to learn the layout options. There are lots.
\geometry{a4paper}                   		% ... or a4paper or a5paper or ... 
%\geometry{landscape}                		% Activate for rotated page geometry
\usepackage[parfill]{parskip}    		% Activate to begin paragraphs with an empty line rather than an indent
\usepackage{graphicx}				% Use pdf, png, jpg, or eps§ with pdflatex; use eps in DVI mode
								% TeX will automatically convert eps --> pdf in pdflatex

\usepackage{hyperref}
		
%SetFonts

\usepackage[T1]{fontenc}
\usepackage[utf8]{inputenc}

\usepackage{tgbonum}

%SetFonts

\title{
	\textbf{
		BIOL 5316
	} \\
	\large Principles of Terrestrial Ecosystem Ecology \\
	\large Spring 2026
}

\date{\vspace{-5ex}}

\begin{document}

{\fontfamily{phv}\selectfont %select helvetica (code = phv)

\maketitle

\section{Course Description}
Students in this course will learn the fundamentals of ecosystem ecology. This will include interactions between biological organisms and themselves as well as their environment. Concepts taught will include water and energy cycling as well as carbon and nutrient flows in natural and managed systems. This will include aboveground and belowground processes. As ecosystem ecology is the largest scale of ecology, the class will also cover necessary concepts of individual, population, and community ecology. Spatial extent of processes will extend to the globe. Temporal extent will extend to millenia. The class will consider applied aspects of global change and human decision making.

Students will be evaluated on their ability to discuss and disseminate course topics. Additionally, as graduate students, you will be evaluated on your ability to help lead the implementation and discussion of multiple topics on course content throughout the semester.

\subsection{Class Time and Location}
Tuesdays and Thursdays 9:30-10:50 \par
Biology Building Room 102


\subsection{Instructor}
Dr. Evan Perkowski \par
Experimental Sciences Building II (ESBII) Room 402 \par
evan.a.perkowski@ttu.edu \par
\textit{Meetings by appointment}

\subsection{Text}
Principles of Terrestrial Ecosystem Ecology (2nd Edition; 2011) by Chapin, Matson, and Vitousek \par
The book can be accessed from Springer here: \url{https://link.springer.com/book/10.1007/978-1-4419-9504-9}. Click on "Access this title on SpringerLink." It can also be accessed through the TTU library.

\section{Statement on connection to and differences from BIOL 4316}
This class is being offered jointly alongside an undergraduate version of the same course, BIOL 4316. The content and mode of instruction will be similar. However, there are a number of differences. Graduate students will also complete the literature review independently, rather than in groups of at least 3 as in 4316 (40\% of total grade; see below). It is expected that this review will not only review the current state of knowledge on the topic, but also produce novel findings, questions, and hypotheses.The content of the review should be of suitable quality for partial inclusion in the introduction of a manuscript or proposal or be submittable for publication on its own. Additionally, graduate students in this course will be expected to serve as a topic co-lead for 2 separate weeks during the course. This involves three assignments per week totaling 30\% of their total grade that is not required for undergraduate students. These assignments are to evaluate graduate students at a higher level. Notably they will need to display knowledge of topics that are sufficient to teach these to their peers in the course. This will require substantial reading and studying on the topics. Together, these differences constitute criteria for evaluation that is above and beyond that for undergraduate students in BIOL 4316. In total, 50\% of the written final literature review (25\% * 50\% = 12.5\%), 70\% of the oral presentation of the literature review (5\% * 70\% = 3.5\%) will have unique evaluation criteria. In addition, 30\% of the graduate student total grade will come from the weekly co-leads, an additional assignment not required by the undergraduate section. The additional evaluation criteria are in place to more critically evaluate graduate student understanding and mastery of the course content.

\section{Course Materials}
All course materials, including lecture slides, readings, activities, and code will be posted to RaiderCanvas.

\section{Learning Objective}
This course will broadly focus on understanding the interactions between biological organisms and their environment that drive cycles of energy, water, carbon, and nutrients at local to global scales. An emphasis will be placed on how these processes influence humanity in a changing world. Applied concepts will consider human decisions as a means to reduce the negative impact of global change. Class activities will be based on discussion and dissemination of ideas,  including classic and recent scientific literature. Topics will be flexible and modified to match student interests where possible.

\section{Attendance Policy}
Attendance is strongly recommended. In class activity points will only be granted if students are in class. Makeups will not be granted.

\section{Course Assessment}
\subsection{\textit{Participation and Engagement}}
Being an active and engaged participant in the class will benefit your understanding of material as well as your peers'. Examples include asking questions, providing feedback, and facilitating discussion.

\subsection{\textit{Mini-quizzes}}
Short “quizzes” will be given each week (typically on Thursdays). These quizzes will be used to assess how well prior concepts were understood by the class. Quizzes will be graded for completion and participation in the ensuing class discussion.

\subsection{\textit{Reading feedback}}
Each week students will be required to read a a section of the book and produce a short summary as well as two questions that arose during their reading. 

\subsection{\textit{Recent literature feedback}}
Students not co-leading the current week’s discussion will be required to produce a summary and develop two questions based on each week’s chosen recent literature article.

\subsection{\textit{Weekly co-leads}}
Twice throughout the semester, students will be asked to co-lead on the week's discussion topic with Dr. Perkowski. This will consist of answering class feedback questions from the reading feedback. It will also consist of leading a 50-minute discussion of a recent literature article of their choice and answering class questions from the Current literature feedbacks. Students will be evaluated on their ability to respond accurately to their peers' questions as well as their ability to summarize and generate discussion on a recent literature article.

\subsection{\textit{Literature Review}}
The primary semester project will be to produce a literature review on a topic of the student's choice. Broadly, the review should address a question or problem related to terrestrial ecosystem ecology and review the current state of knowledge on the topic. The review should be forward thinking, in that it forms the basis for understanding ecosystem moving forward. The review should be novel in that it should not be similar to previously published review papers. \textbf{It is highly encouraged that students choose a topic that is related to their research interests that could be used as an initial chapter for their thesis or dissertation.}

Students will develop a written proposal for their literature review and present their idea to the class. The class and instructors will provide feedback. Students will then produce and present their review to the class at the end of the semester. 

This project will be done individually. Students are encouraged to receive help and guidance from the instructors as well as the class at large, particularly at the idea generation phase of the project.

The literature review will be assessed for completeness, breadth, originality, and presentation. Students must have their project approved by the instructor after the proposal and prior to beginning the final project.

The grading of this review will differ from that for the undergraduate section. Specifically,for full points on the breadth and originality portion (20\% of written grade), the question addressed must be completely novel andnot something that has been reviewed in the published literature. The review should produce novel findings, questions, and hypotheses. For full points on the scientific rigor portion (30\% of written grade), the content of the review should be of suitable quality for partial inclusion in the introduction of a manuscript or proposal or be submittable for publication on its own. For full points on the oral presentation portion (5\% of total oral grade), publication on its own. For full points on the oral presentation portion (5\% of total oral grade), the presentation should be of suitable quality that it could be presented at a national conference in the field of plant science (Botanical Society of America) and/or ecology (Ecological Society of America).

\newpage

\section{Grading}
Participation and Engagement: 20\% \par
Mini-quizzes: 5\% \par
Reading feedback: 5\% \par
Co-lead discussion leads: 10\% \par
Co-lead response to reading feedback: 10\% \par
Co-lead response to recent literature feedback: 10\% \par
Recent literature feedback: 5\% \par
Literature review idea proposal: 10\% \par
Final literature review presentation: 5\% \par
Final literature review: 25\% \par

Grades will be made available on RaiderCanvas. 
All grades posted at the end of the course will be final, unless an error has been made in their calculation. Please contact Dr. Perkowski if you feel your grade has been calculated incorrectly.

\section{Grading Scale}
A: $\geq$ 90\% \par
B: 80 – 90\% \par
C: 70 – 80\% \par
D: 60 – 70\% \par
F: $\leq$ 59.9\% \par

\newpage
\section*{Topic Schedule by Week}
 January 15, 2026 - No class Thursday \par
 January 20, 2026 - Introductions, semester planning, and the Ecosystem Concept (Ch. 1) \par
 January 27, 2026 - The Ecosystem Concept (continued; Ch. 1) \par
 February 3, 2026 - Climate and Soils (Ch. 2, 3) \par
 February 10, 2026 - Water and Energy Balance (Ch. 4); \textbf{Literature review idea list due} \par
 February 17, 2026 - Carbon Inputs (Ch. 5) \par
 February 24, 2026 - Ecosystem carbon budgets (Ch. 7) \par
 March 2, 2026 - Plant Nutrient Use and Nutrient Cycling (Ch. 8,9); \textbf{Literature review idea proposals due} \par
 March 9, 2026 - Species and Trophic Dynamics (Ch. 10, 11) \par
 March 16, 2026 - \textbf{No class - Spring Break} \par
 March 23, 2026 - Temporal Dynamics (Ch. 12) \par
 March 30, 2026 - Landscape heterogeneity and ecosystem dynamics (Ch. 13) \par
 April 6, 2026 - Changes in the Earth system (Ch. 15); \textbf{Literature review rough drafts due} \par
 April 13, 2026 - Managing and sustaining ecosystems (Ch. 15) \par
 April 20, 2026 - Literature review presentations \par
 April 27, 2026 - Literature review presentations; \textbf{Literature review final drafts due} \par

 \section*{Deadlines (due11:59pm on RaiderCanvas)}
 March 5, 2026 - \textbf{Literature review idea proposal} \par
 April 6, 2026 - \textbf{Literature review rough draft} \par
 April 29, 2026 - \textbf{Literature review final draft} \par

} %end font selection

\end{document} 
