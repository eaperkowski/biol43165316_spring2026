\documentclass[12pt, notitlepage]{article}   	% use "amsart" instead of "article" for AMSLaTeX format
\usepackage{geometry}                		% See geometry.pdf to learn the layout options. There are lots.
\geometry{a4paper}                   		% ... or a4paper or a5paper or ... 
%\geometry{landscape}                		% Activate for rotated page geometry
\usepackage[parfill]{parskip}    		% Activate to begin paragraphs with an empty line rather than an indent
\usepackage{graphicx}				% Use pdf, png, jpg, or eps§ with pdflatex; use eps in DVI mode
								% TeX will automatically convert eps --> pdf in pdflatex

\usepackage{hyperref}
		
%SetFonts

\usepackage[T1]{fontenc}
\usepackage[utf8]{inputenc}

\usepackage{tgbonum}

%SetFonts

\title{
	\textbf{
		BIOL 4316
	} \\
	\large Principles of Terrestrial Ecosystem Ecology \\
	\large Spring 2026
}

\date{\vspace{-5ex}}

\begin{document}

{\fontfamily{phv}\selectfont %select helvetica (code = phv)

\maketitle

\section{Course Description}
Students in this course will learn the fundamentals of ecosystem ecology. This will include interactions between biological organisms and themselves as well as their environment. Concepts taught will include water and energy cycling as well as carbon and nutrient flows in natural and managed systems. This will include aboveground and belowground processes. As ecosystem ecology is the largest scale of ecology, the class will also cover necessary concepts of individual, population, and community ecology. Spatial extent of processes will extend to the globe. Temporal extent will extend to millenia. The class will consider applied aspects of global change and human decision making. 

Students will be evaluated on their ability to discuss and disseminate course topics.

\subsection{Class Time and Location}
Tuesdays and Thursdays 9:30-10:50 \par
Biology Building Room 102

\subsection{Instructor}
Dr. Evan Perkowski \par
Experimental Sciences Building II (ESBII) Room 402 \par
evan.a.perkowski@ttu.edu \par
\textit{Meetings by appointment}

\subsection{Text}
Principles of Terrestrial Ecosystem Ecology (2nd Edition; 2011) by Chapin, Matson, and Vitousek \par
The book can be accessed from Springer here: \url{https://link.springer.com/book/10.1007/978-1-4419-9504-9}. Click on "Access this title on SpringerLink." It can also be accessed through the TTU library.

\section{Course Materials}
All course materials, including lecture slides, readings, activities, and code will be posted to RaiderCanvas.

\section{Learning Objective}
This course will broadly focus on understanding the interactions between biological organisms and their environment that drive cycles of energy, water, carbon, and nutrients at local to global scales. An emphasis will be placed on how these processes influence humanity in a changing world. Applied concepts will consider human decisions as a means to reduce the negative impact of global change. Class activities will be based on discussion and dissemination of ideas, including classic and recent scientific literature. Topics will be flexible and modified to match student interests where possible. 

\section{Attendance Policy}
Attendance is strongly recommended. In-class points will only be granted if students are in class. Makeups will not be granted.

\section{Course Assessment}
\subsection{\textit{Participation and Engagement}}
Being an active and engaged participant in the class will benefit your understanding of material as well as your peers'. Examples include asking questions, providing feedback, and facilitating discussion.

\subsection{\textit{Mini-quizzes}}
Short “quizzes” will be given each week (typically on Thursdays). These quizzes will be used to assess how well prior concepts were understood by the class. Quizzes will be graded for completion and participation in the ensuing class discussion.

\subsection{\textit{Reading feedback}}
Each week students will be required to read a a section of the book and produce a short summary as well as two questions that arose during their reading. 

\subsection{\textit{Recent literature feedback}}
Students will be required to produce a summary and develop two questions based on each week’s chosen recent literature article.

\subsection{\textit{Literature review}}
The primary semester project will be to produce a literature review on a topic of the student's choice. Broadly, the review should address a question or problem related to terrestrial ecosystem ecology and review the current state of knowledge on the topic. The review should be forward thinking, in that it forms the basis for understanding ecosystem dynamics moving forward. The review should be novel in that it should not be similar to previously published review papers.

Students will first develop a written proposal for their literature review. Dr. Perkowski will provide feedback. Students will then produce and present their review to the class at the end of the semester. 

This project will be done in groups of at least 3 people. Students are encouraged to receive help and guidance from the instructors as well as the class at large. 

The literature review will be assessed for completeness, breadth, originality, and presentation. Students must have their project approved by the instructor after the proposal and prior to beginning the final project.

\section{Grading}
Participation and Engagement: 25\% \par
Mini-quizzes: 15\% \par
Reading feedback: 10\% \par
Recent literature feedback: 10\% \par
Literature review idea proposal: 15\% \par
Final literature review presentation: 10\% \par
Final literature review: 25\% \par

Grades will be made available on RaiderCanvas. All grades posted at the end of the course will be final, unless an error has been made in their calculation.
Please contact Dr. Perkowski if you feel your grade has been calculated incorrectly.

\section{Grading Scale}
A: $\geq$ 90\% \par
B: 80 – 90\% \par
C: 70 – 80\% \par
D: 60 – 70\% \par
F: $\leq$ 59.9\% \par

\section{Academic Integrity Statement}
Academic integrity is taking responsibility for one’s own class and/or course work, being individually accountable, and demonstrating intellectual honesty and ethical behavior. Academic integrity is a personal choice to abide by the standards of intellectual honesty and responsibility. Because education is a shared effort to achieve learning through the exchange of ideas, students, faculty, and staff have the collective responsibility to build mutual trust and respect. Ethical behavior and independent thought are essential for the highest level of academic achievement, which then must be measured. Academic achievement includes scholarship, teaching, and learning, all of which are shared endeavors. Grades are a device used to quantify the successful accumulation of knowledge through learning. Adhering to the standards of academic integrity ensures grades are earned honestly. Academic integrity is the foundation upon which students, faculty, and staff build their educational and professional careers [Texas Tech University Quality Enhancement Plan, Academic Integrity Task Force, 2010].

\section{Plagiarism Statement}
Texas Tech University expects students to “understand the principles of academic integrity and abide by them in all class and/or course work at the University” (OP 34.12.5). Plagiarism is a form of academic misconduct that involves (1) the representation of words,  ideas, illustrations, structure, computer code, other expression, or media of another as one's own and/or failing to properly cite direct, paraphrased, or summarized materials; or (2) self-plagiarism, which involves the submission of the same academic work more than once without the prior permission of the instructor and/or failure to correctly cite previous work written by the same student. Please review Section B of the TTU Student Handbook for more information related to other forms of academic misconduct, and contact your instructor if you have questions about plagiarism or other academic concerns in your courses. To learn more about the importance of academic integrity and practical tips for avoiding plagiarism, explore the resources provided by the TTU Library and the School of Law.

\section{AI Use}
The use of generative AI tools (such as ChatGPT) is strictly prohibited in this course for any purpose. Information gathered from AI cannot be used even with appropriate citation. Submission of AI-generated content (i.e., information, text, or images) as your own work is a violation of academic integrity and may result in referral to the Office of Student Conduct. Please contact Dr. Perkowski if you have questions regarding this course policy.

\section{ADA Statement}
Any student who, because of a disability, may require special arrangements in order to meet the course requirements should contact the instructor as soon as possible to make any necessary arrangements. Students should present appropriate verification from Student Disability Services during the instructor’s office hours. Please note: instructors are not allowed to provide classroom accommodations to a student until appropriate verification from Student Disability Services has been provided. For additional information, please contact Student Disability Services in Weeks Hall or call 806-742-2405.

\section{Religious Holy Day Statement}
"Religious holy day" means a holy day observed by a religion whose places of worship are exempt from property taxation under Texas Tax Code §11.20. A student who intends to observe a religious holy day should make that intention known in writing to the instructor prior to the absence. A student who is absent from classes for the observance of a religious holy day shall be allowed to take an examination or complete an assignment scheduled for that day within a reasonable time after the absence. A student who is excused under section 2 may not be penalized for the absence; however, the instructor may respond appropriately if the student fails to complete the assignment satisfactorily.

\section{Discrimination, Harassment, and Sexual Violence Statement}
Texas Tech University is committed to providing and strengthening an educational, working, and living environment where students, faculty, staff, and visitors are free from gender and/or sex discrimination of any kind. Sexual assault, discrimination, harassment, and other Title IX violations are not tolerated by the University. Report any incidents to the Office for Student Rights \& Resolution, (806)-742-SAFE (7233) or file a report online through the Title IX office. Faculty and staff members at TTU are committed to connecting you to resources on campus. Some of these available resources are:
\begin{itemize}
    \item \textbf{TTU Student Counseling Center}, 806- 742-3674: Provides confidential support on campus
    \item \textbf{TTU 24-hour Crisis Helpline}, 806-742-5555: Assists students who are experiencing a mental health or interpersonal violence crisis. If you call the helpline, you will speak with a mental health counselor.
    \item \textbf{Voice of Hope}, 806-763-7273: 24-hour hotline that provides support for survivors of sexual violence.
    \item \textbf{Risk, Intervention, Safety and Education (RISE) Office}, 806-742-2110: Provides a range of resources and support options focused on prevention education and student wellness.
    \item \textbf{Texas Tech Police Department}, 806-742- 3931: To report criminal activity that occurs on or near Texas Tech campus.
\end{itemize}

\section{Classroom Civility}
Texas Tech University is a community of faculty, students, and staff that enjoys an expectation of cooperation, professionalism, and civility during the conduct of all forms of university business, including the conduct of student–student and student–faculty interactions in and out of the classroom. Further, the classroom is a setting in which an exchange of ideas and creative thinking should be encouraged and where intellectual growth and development are fostered. Students who disrupt this classroom mission by rude, sarcastic, threatening, abusive or obscene language and/or behavior will be subject to appropriate sanctions according to university policy. Likewise, faculty members are expected to maintain the highest standards of professionalism in all interactions with all constituents of the university \newline (\href{www.depts.ttu.edu/ethics/matadorchallenge/ethicalprinciples.php}{www.depts.ttu.edu/ethics/matadorchallenge/ethicalprinciples.php}).

\section{Student Support}
The Office of Campus Access and Engagement works across Texas Tech University to foster, affirm, celebrate, engage, and strengthen all student communities. For more information about services, opportunities for participation, and ways in which Texas Tech can support your success in college, please contact (806) 742-7025.

\section{Food Insecurity}
If you are a student experiencing food or housing insecurity, and you believe this may affect your performance in the course, I strongly encourage you to reach out to the Raider Relief Advocacy and Resource Center for support. The Raider Relief Advocacy and Resource Center (RR-ARC) offers a central hub for students facing challenges related to basic needs. Through a broad network of campus and community resources, we aim to reduce the impact of financial, physical, and emotional difficulties, helping students thrive academically and personally. Units within the RR-ARC include Raider Relief Fund, Raider Red's Food Pantry (Doak Hall 117), and Fostering the Future. If you need assistance, please reach out to get support.

\section{Statement of Accommodation for Pregnant Students}
To support the academic success of pregnant and parenting students and students with pregnancy related conditions, the University offers reasonable modifications based on the student’s particular needs. Any student who is pregnant or parenting a child up to age 18 or has conditions related to pregnancy may contact Alex Faris, the Texas Tech University designated Pregnancy and Parenting Liaison, to discuss support available through the University. The Liaison can be reached by emailing \href{mailto:alfaris@ttu.edu}{alfaris@ttu.edu}. Should a student communicate with the instructor that they are pregnant or have a pregnancy related condition or may need additional resources related to pregnancy or parenting, the instructor will communicate that student’s information to the Title IX Coordinator, who will work with the student and others, as needed, to ensure equal access to the University’s education program or activity. 

For more information regarding supportive measures, please contact pregnancy \& parenting liaison Alex Faris (\href{mailto:alfaris@ttu.edu}{alfaris@ttu.edu} | 806.834.3420) or visit \newline \href{https://www.depts.ttu.edu/titleix/PregnancyandParenting/}{https://www.depts.ttu.edu/titleix/PregnancyandParenting/}. You can also visit \newline \href{https://www.depts.ttu.edu/titleix/PregnancyandParenting/}{https://www.depts.ttu.edu/titleix/PregnancyandParenting/} to submit a request to Alex Faris for assistance.

\newpage
\section*{Topic Schedule by Week}
 January 15, 2026 - No class Thursday \par
 January 20, 2026 - Introductions, semester planning, and the Ecosystem Concept (Ch. 1) \par
 January 27, 2026 - The Ecosystem Concept (continued; Ch. 1) \par
 February 3, 2026 - Climate and Soils (Ch. 2, 3) \par
 February 10, 2026 - Water and Energy Balance (Ch. 4); \textbf{Literature review idea list due} \par
 February 17, 2026 - Carbon Inputs (Ch. 5) \par
 February 24, 2026 - Ecosystem carbon budgets (Ch. 7) \par
 March 2, 2026 - Plant Nutrient Use and Nutrient Cycling (Ch. 8,9); \textbf{Literature review idea proposals due} \par
 March 9, 2026 - Species and Trophic Dynamics (Ch. 10, 11) \par
 March 16, 2026 - \textbf{No class - Spring Break} \par
 March 23, 2026 - Temporal Dynamics (Ch. 12) \par
 March 30, 2026 - Landscape heterogeneity and ecosystem dynamics (Ch. 13) \par
 April 6, 2026 - Changes in the Earth system (Ch. 15); \textbf{Literature review rough drafts due} \par
 April 13, 2026 - Managing and sustaining ecosystems (Ch. 15) \par
 April 20, 2026 - Literature review presentations \par
 April 27, 2026 - Literature review presentations; \textbf{Literature review final drafts due} \par

 \section*{Deadlines (due11:59pm on RaiderCanvas)}
 February 10, 2026 - \textbf{Literature review idea list} \par
 March 5, 2026 - \textbf{Literature review idea proposal} \par
 April 6, 2026 - \textbf{Literature review rough draft} \par
 April 29, 2026 - \textbf{Literature review final draft} \par

} %end font selection

\end{document} 
